\section{Summary functions}

An image summary function (embedding) $\summ$ is a continuous function that maps an image of a material ($\target$ or $\synth$) into a vector in $\Reals^k$. An idealized summary function would have the property that
\begin{equation}
	\summ(f(\btheta_1, \bz_1)) = \summ(f(\btheta_2, \bz_2)) \ \Leftrightarrow \ \btheta_1 = \btheta_2.
\end{equation}
That is, applying the summary function would fully abstract away from the randomness $\bz$ and the difference between the two summary vectors would be entirely due to different material properties $\btheta$.

Practical summary functions will satisfy the above only approximately. However, a good practical summary function will embed images of the same material close to each other, and images of different materials further away from each other. In that case, we can pose the parameter estimation problem as follows:
\begin{equation}
	\mbox{minimize} \ d(\target, f(\btheta, \bz)),
\end{equation}
where $d$ is a suitable distance metric on $\Reals^k$, for example a (weighted) L2-norm. Below we discuss several techniques for constructing summary functions.



\subsection{Statistics of image bins}

The simplest idea for a summary function is to subdivide the image into $k$ bins (regions) and compute the (scalar or RGB) mean of each region. For mostly isotropic materials, a flash photograph will normally lead to an approximately radially symmetric highlight, where we found concentric bins perform well. For anisotropic highlights (e.g. brushed metal), it makes more sense to define the bins as $k$ vertical or horizontal bands.

This simple approach can already be quite effective in some cases (e.g. homogeneous scattering media). However, adding the standard deviation statistic can significantly help with estimating frequencies and size of features such as bumps.


\subsection{Fourier transforms}

However, the simple per-bin statistics can be insufficient to precisely match frequency characteristics of material imperfections. A more complex and more powerful tool for summary function design is fast Fourier transforms, which captures frequencies explicitly. These can be applied to the whole image (as a 2D FFT), or part of it, e.g. as 1D FFTs over a subset of rows or columns. Note that automatic computation of derivatives is possible with the Fourier transform, and supported by the \torch framework.

For some material models, we find that the means of concentric bins in the original as well as Fourier-transformed image provide an effective summary function.


% \subsection{Learned summary functions}

% In addition to hand-designed summary functions, we have also considered learning them automatically in the form of convolutional neural networks. TODO

