Procedural material models have been graining traction in many applications thanks to their flexibility, compactness, and easy editability.
In this paper, we explore the inverse rendering problem of procedural material parameter estimation from photographs using a Bayesian framework.
We use \emph{summary functions} for comparing unregistered images of a material under known lighting, and we explore both hand-designed and neural summary functions. In addition to estimating the parameters by optimization, we introduce a Bayesian inference approach using Hamiltonian Monte Carlo to sample the space of plausible material parameters, providing additional insight into the structure of the solution space.
To demonstrate the effectiveness of our techniques, we fit procedural models of a range of materials---wall plaster, leather, wood, anisotropic brushed metals and metallic paints---to both synthetic and real target images.
%and scattering media (e.g. milk). \sz{milk is not really a procedural material...}