\subsection{Translucent materials}
\label{ssec:scattering}
%
Translucent materials allow light to penetrate their surface and scatter within the interior.
Our forward model focuses on semi-infinite and homogeneous media with a Henyey-Greenstein (HG) phase function lit by a point light source depicted using the following parameters:
%
\[
\btheta = (\sigma_s, \sigma_a, g, \light),
\]
%
where $\sigma_s$ and $\sigma_a$ are respectively the material's scattering and absorption coefficients, $g$ indicates the phase function's first Legendre moment (average cosine), and $\light$ is the light intensity.
Further, we assume that the medium's refractive index $\eta$ is known.

To render these materials, we use volumetric path tracing (VPT) with next-event estimation (NEE).
To handle mismatching refractive indices (i.e., $\eta \neq 1$), we leverage a specialized NEE strategy similar to the one introduced by Walter~et~al.~\cite{Walter:2009:SSR}.
Specifically, given an interior scattering location $\bx$ and the light location $\by$, we search for the refraction point $\bx'$ on the interface such that the light path $\by \to \bx' \to \bx$ satisfies Snell's law (at $\bx'$).
To find this point, we analytically solve a quartic function, which is guaranteed to have exactly one root corresponding to $\bx'$.
% \sz{Details?}

After obtaining $\bx'$, our modified NEE process tallies
%
\begin{multline}
	\label{eqn:NEE}
	F_t(\by \to \bx' \to \bx) \, f_p(\bx \to \bx', \bom) \, \exp\left(-(\sigma_s + \sigma_a) \| \bx - \bx' \|\right)\\
	\left[ (D_1 + \eta D_2) \left( \tfrac{\cos\theta_2}{\cos\theta_1}D_1 + \tfrac{\cos\theta_1}{\cos\theta_2} \eta D_2 \right) \right]^{-1},
\end{multline}
%
where $F_t$ denotes the Fresnel transmission term, $f_p$ is the HG phase function, and the exponential term captures the transmittance between $\bx$ and $\bx'$.
Further, the last term in Eq.~\eqref{eqn:NEE} accounts for the change of measure from location to solid angle with refraction factored in.
In this term, $D_1 := \| \bx - \bx' \|$, $D_2 := \| \bx' - \by \|$, and $\theta_1$, $\theta_2$ are the angles from $\bx' \to \bx$ and $\bx \to \by$ to the surface normal, respectively. % (see Figure~\ref{fig:NEE}).
Note that, when $\eta = 1$, this term reduces to $\| \bx - \by \|^{-2}$, which is precisely the intensity falloff factor for point sources.

% \begin{figure}[b]
% 	\centering
% 	\includegraphics[height=1in]{placeholder/placeholder.jpg}
% 	\caption{\label{fig:NEE}
% 		An illustration of our extended next-event estimation process.
% 	}
% \end{figure}

As the calculation of $\bx'$ and Eq.~\eqref{eqn:NEE} are both analytical, our forward process can be differentiated symbolically with respect to all the material parameters $\btheta$.
Further, as the appearance of semi-infinite homogeneous media is generally very smooth and symmetric around the projection of the light source, we use average pixel intensities within radically symmetric bins (similar to the case for bumpy materials) as the summary function. Note that this entire forward simulation is still implemented using array-level operations in PyTorch, like in previous examples; the Monte Carlo scattering path sampling in the simulation is distinct from (and serves as the inner loop of) the Hamiltonian Monte Carlo sampling of the posterior.

\paragraph{Similarity relations}
In radiative transfer material parameter space (i.e., $\sigma_s$, $\sigma_a$, and $g$) is known to be (approximately) over-complete, especially with the absence of sharp geometries (which applies in our flat configuration).
That is, multiple combinations of these parameters can yield roughly identical appearances.
This effect is mathematically captured by similarity relations~\cite{Zhao:2014:HSR}.
Specifically, the first-order variant of the similarity relations states that two scattering media with parameters $(\sigma_s, \sigma_a, g)$ and $(\sigma_s^*, \sigma_a^*, g^*)$ will have approximately identical appearances if
%
\begin{equation}
\sigma_a = \sigma_a^*, \quad
\sigma_s (1 - g) = \sigma_s^* (1 - g^*).
\end{equation}

The presence of similarity relations has been a challenge for solving inverse scattering problems because of the fundamental difficulty in distinguishing parameters within the same similarity class.
Our technique, which provides posterior distributions rather than single estimates, is capable of automatically detecting such structures.
As shown in Figure~\ref{fig:scattering1}, the posterior distributions provided by our method match the similarity theory predictions very well.

\begin{figure}[t]
	\centering
	\addtolength{\tabcolsep}{-3pt}
	\begin{tabular}{cc}
		\includegraphics[width=0.45\columnwidth]{results/scatter/synthetic/posterior1.pdf} &
		\includegraphics[width=0.45\columnwidth]{results/scatter/synthetic/posterior2.pdf}
	\end{tabular}
	\caption{\label{fig:scattering1}
		Posterior distributions sampled with our method for two synthetic input images.
		Unlike optimization-based methods which usually converge to some arbitrary locations near the similarity curve, our technique is able to detect the full structure of these ``areas of confusion''.
	}
\end{figure}
