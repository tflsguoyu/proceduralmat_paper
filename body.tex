% ---------------------------------------------------------------------
% EG author guidelines plus sample file for EG publication using LaTeX2e input
% D.Fellner, v2.03, Dec 14, 2018


%\title[EG \LaTeX\ Author Guidelines]%
%      {\LaTeX\ Author Guidelines for EUROGRAPHICS Proceedings Manuscripts}
%
%% for anonymous conference submission please enter your SUBMISSION ID
%% instead of the author's name (and leave the affiliation blank) !!
%% for final version: please provide your *own* ORCID in the brackets following \orcid; see https://orcid.org/ for more details.
%\author[D. Fellner \& S. Behnke]
%{\parbox{\textwidth}{\centering D.\,W. Fellner\thanks{Chairman Eurographics Publications Board}$^{1,2}$\orcid{0000-0001-7756-0901}
%        and S. Behnke$^{2}$\orcid{0000-0001-5923-423X} 
%%        S. Spencer$^2$\thanks{Chairman Siggraph Publications Board}
%        }
%        \\
%% For Computer Graphics Forum: Please use the abbreviation of your first name.
%{\parbox{\textwidth}{\centering $^1$TU Darmstadt \& Fraunhofer IGD, Germany\\
%         $^2$Graz University of Technology, Institute of Computer Graphics and Knowledge Visualization, Austria
%%        $^2$ Another Department to illustrate the use in papers from authors
%%             with different affiliations
%       }
%}
%}
\title[Bayesian Inference]%
      {A Bayesian Inference Framework for \\ Procedural Material Parameter Estimation}

\author[Submission ID: 1000]
{\parbox{\textwidth}{\centering Submission ID: 1000
        }
        \\
{\parbox{\textwidth}{\centering 
       }
}
}
% ------------------------------------------------------------------------

% if the Editors-in-Chief have given you the data, you may uncomment
% the following five lines and insert it here
%
% \volume{36}   % the volume in which the issue will be published;
% \issue{1}     % the issue number of the publication
% \pStartPage{1}      % set starting page

\newcommand{\Reals}{\mathbb{R}}
\newcommand{\summ}{\mathbb{S}}
\newcommand{\la}{\leftarrow}
\newcommand{\ra}{\rightarrow}
\newcommand{\lra}{\leftrightarrow}
\newcommand{\bd}{\bm{d}}
\newcommand{\bz}{\bm{z}}
\newcommand{\btheta}{\bm{\theta}}
\newcommand{\target}{\bm{I}_t}
\newcommand{\synth}{\bm{I}_s}
\newcommand{\bsigma}{\bm{\sigma}}
\newcommand{\fsigma}{\sigma_f}
\newcommand{\fsigmax}{\sigma_{fx}}
\newcommand{\fsigmay}{\sigma_{fy}}
\newcommand{\fscale}{c_f}
\newcommand{\light}{E}
\newcommand{\bx}{\bm{x}}
\newcommand{\by}{\bm{y}}
\newcommand{\bom}{\bm{\omega}}

\newcommand{\milos}[1]{\textcolor{red}{[\textbf{Milos:} {\em #1}]}}
\newcommand{\sz}[1]{\textcolor{blue}{[\textbf{SZ:} {\em #1}]}}
\newcommand{\revision}[1]{#1}

\newcommand{\torch}{\texttt{libtorch} \xspace}

\DeclareMathOperator*{\argmin}{arg\,min}
\DeclareMathOperator*{\argmax}{arg\,max}

\newlength{\resultwidth}
\setlength{\resultwidth}{1in}

%-------------------------------------------------------------------------
\begin{document}

% uncomment for using teaser
% \teaser{
%  \includegraphics[width=\linewidth]{eg_new}
%  \centering
%   \caption{New EG Logo}
% \label{fig:teaser}
%}

\maketitle
%-------------------------------------------------------------------------
\begin{abstract}
Procedural material models have been graining traction in many applications thanks to their flexibility, compactness, and easy editability.
In this paper, we explore the inverse rendering problem of procedural material parameter estimation from photographs using a Bayesian framework.
We use \emph{summary functions} for comparing unregistered images of a material under known lighting, and we explore both hand-designed and neural summary functions. In addition to estimating the parameters by optimization, we introduce a Bayesian inference approach using Hamiltonian Monte Carlo to sample the space of plausible material parameters, providing additional insight into the structure of the solution space.
To demonstrate the effectiveness of our techniques, we fit procedural models of a range of materials---wall plaster, leather, wood, anisotropic brushed metals and metallic paints---to both synthetic and real target images.
%and scattering media (e.g. milk). \sz{milk is not really a procedural material...}
%-------------------------------------------------------------------------
%  ACM CCS 1998
%  (see https://www.acm.org/publications/computing-classification-system/1998)
% \begin{classification} % according to https://www.acm.org/publications/computing-classification-system/1998
% \CCScat{Computer Graphics}{I.3.3}{Picture/Image Generation}{Line and curve generation}
% \end{classification}
%-------------------------------------------------------------------------
%  ACM CCS 2012
   (see https://www.acm.org/publications/class-2012)
%The tool at \url{http://dl.acm.org/ccs.cfm} can be used to generate
% CCS codes.
%Example:
\begin{CCSXML}
	<ccs2012>
	<concept>
	<concept_id>10010147.10010371.10010372.10010376</concept_id>
	<concept_desc>Computing methodologies~Reflectance modeling</concept_desc>
	<concept_significance>500</concept_significance>
	</concept>
	</ccs2012>
\end{CCSXML}

\ccsdesc[500]{Computing methodologies~Reflectance modeling}

\printccsdesc   
\end{abstract}  
%-------------------------------------------------------------------------
\section{Introduction}
\label{sec:intro}
%
Physically accurate simulation of material appearance is an important yet challenging problem, with applications in many areas from entertainment to product design and architecture visualization.
A key ingredient to photorealistic rendering is high-quality material models.
Acquiring the parameters of these models from physical measurements (for example, photographs) has been an active research topic in computer vision and graphics.

Recently, \emph{procedural} material models have been gaining significant traction in the industry (e.g., \cite{Substance}).
In contrast to traditional material reflectance models that represent spatially varying surface albedo, roughness, and normal vectors as 2D images, the procedural models generate such information using a smaller number of user-facing parameters, providing high compactness and easy editability.

In this paper, we introduce a new differentiable computational framework to estimate the parameters of procedural material models.
Our technique enjoys generality by covering a range of materials from standard opaque dielectrics (e.g. plastics, leather, wall paint, wood) to anisotropic brushed metals and metallic paints (Figure~\ref{fig:teaser}). %to scattering media (e.g. milk).
Further, we introduce a novel view of the procedural parameter estimation problem in a \emph{Bayesian framework}, precisely defining prior and posterior distributions of the parameters, allowing for both maximization and sampling of the posterior.

%Our focus on procedural materials is in contrast to several recent methods that estimate parameters separately per pixel. These methods make the assumption of a simple BRDF model (a microfacet specular term, a diffuse term, and a varying normal vector); and their result is in the form of several textures (commonly including diffuse albedo, normal, roughness, and specular coefficient). While the results of the recent methods are impressive, the reality is that procedural materials are gaining significant traction in the industry \cite{Substance}, due to their ability to cover large areas without repetition, easy editability, and small storage requirement.

 \begin{figure*}[t]
 	\centering
 	\includegraphics[width=\textwidth]{images/img/teaser.png}
 	\caption{\label{fig:teaser}
 		A scene rendered with material parameters estimated using our method: bumpy dielectrics, leather, plaster, wood, brushed metal, and metallic paint. The insets show a few examples of the initial flash photograph, and our procedural material with parameters found by posterior maximization.
 	}
 \end{figure*}

The estimation of procedural model parameters faces several challenges. First, the procedural generation (and physics-based rendering) of materials is a complex process with a diverse set of operations, making the relationship between procedural model parameters and properties of the final renderings nonlinear and complicated.
Additionally, designing a suitable \emph{loss function} (metric) to compare a synthesized image to a target image is not obvious.
This is because the procedurally generated images do not offer pixel-wise alignments to target images, making simple image difference metrics (e.g., L2 or SSIM) unsuitable.

Our contributions lie in the following two main areas. First, we present a general framework to %address the loss function design problem
compare simulated and target images in a robust fashion without requiring pixel-wise alignments~(\S\ref{sec:summary_func}). To this end, we leverage \emph{summary functions} that map images to latent vectors which can then be compared using L2 or other simple metrics to judge how different two images are.
%A well-designed summary function can be applied to both the simulated and the target image, after which the L2-norm or similar simple metrics can be used to judge the error of a fit.
We consider a number of summary functions, from very simple ones (means of fixed image regions), through higher order statistics and Fourier transforms, to neural summary functions (embeddings) based on Gram matrices of VGG feature maps \cite{Gatys2015,Gatys2016}. The neural embedding approach was first introduced to material capture by Aittala et al. \cite{Aittala2016}; we extend their approach to the case of procedural materials, were it turns out to perform well.

Second, we introduce a \emph{Bayesian inference} approach using Hamiltonian Monte Carlo (HMC) sampling of the space of plausible material parameters~(\S\ref{sec:bayesian}). This provides additional information beyond single point estimates of material parameters (for example, though not limited to, discovering similarity structures in the parameter space). Posterior sampling is a well-studied area within statistics but, to our knowledge, has not yet been applied to material appearance acquisition (or inverse rendering in general).

We implement the procedural generation and rendering processes as (differentiable) PyTorch procedures with the priors and summary functions (classical or neural) expressed in the same framework. These four components (priors, procedural material model, rendering, summary function) together define our posterior distribution (outlined in Figure~\ref{fig:posterior}), a rather complex (but fully differentiable) function.

To demonstrate the efficacy of our techniques, we fit procedural models of a few materials to both synthetic and real target images (\S\ref{sec:results}).

\begin{figure*}[t]
	\includegraphics[width=\textwidth]{images/img/posterior.pdf}
	\caption{Our differentiable posterior computation combines priors, a procedural material model, a rendering operator, a summary function, and a target image.}
	\label{fig:posterior}
\end{figure*}

\section{Related work}

Here we cover previous work on material parameter estimation in computer graphics, as well as on Hamiltonian Monte Carlo methods in Bayesian inference.

\paragraph{Per-pixel SVBRDF capture.} A large amount of previous work in computer graphics focuses on acquisition of materials from physical measurements. The methods generally observe the material sample with a fixed camera position, and solve for the parameters of a BRDF model such as diffuse albedo, roughness (glossiness) and surface normal. They differ in the number of light patterns required and their type; the patterns used include moving linear light \cite{Gardner2003}, Gray code patterns \cite{Francken2009} and spherical harmonic illumination \cite{Ghosh2009}. In these approaches, the model and its optimization are specific to the light patterns and the optical setup of the method, as general non-linear optimization was historically deemed too inefficient and not robust enough.

More recently, Aittala et al. \shortcite{Aittala2013} captured per-pixel SVBRDF data using Fourier patterns projected using an LCD screen; their algorithm used a fairly general, differentiable forward evaluation model, which was inverted in a maximum a-posteriori (MAP) framework. In practice, this was done using a standard non-linear least-squares optimizer with well-chosen priors, showing that a general optimization approach with a differentiable forward model can be successful with carefully chosen priors and initialization.

Later work by Aittala et al. \shortcite{Aittala2015,Aittala2016} found per-pixel parameters of stationary spatially-varying SVBRDFs from two-shot and one-shot flash-lit photographs, respectively. In the latter case, the approach used a neural texture descriptor as a more advanced loss function without requiring per-pixel alignment; this is related to our summary function concept.

Recent methods by Deschaintre et al. \shortcite{Deschaintre2018} and Li et al. \shortcite{Li2018} have been able to capture non-stationary SVBRDFs from a single flash photograph by training an end-to-end deep convolutional network. All of these approaches estimate per-pixel parameters of the microfacet-diffuse-normal model, and are not obviously applicable to estimation of global model parameters, nor to more advanced optical models used in some of our examples (significant anisotropy, layering or scattering).

\paragraph{Global parameter estimation.} Focus on estimating the global parameters of material models (that is, not per-pixel estimates) has been relatively rare in previous material capture efforts; however, there are a few exceptions. The dual-scale glossy parameter estimation work of Wang et al. \shortcite{Wang2011} finds, under step-edge lighting, the parameters of a bumpy surface model consisting of a heightfield constructed from a Gaussian noise power spectrum and global microfacet material parameters. Their results provide impressive accuracy, but their solution is highly specialized for this material model and illumination. This problem is closely related to our ``bumpy'' example, which essentially replicates their result (under flash illumination), but using a more general framework.

Several approaches for appearance rendering of cloth have used micro-CT scans to model the material at the microscopic fiber level. However, the optical properties of the fibers (e.g. roughness, scattering albedo) are not available from the scans and have to be chosen separately. Zhao et al. \shortcite{Zhao2011} use a simple but effective trick of matching the mean and standard deviation (in RGB) of the pixels in a well-chosen area of the target and simulated image. Khungurn et al. \shortcite{Khungurn2015} have extended this approach with a differentiable volumetric renderer, combined with a stochastic gradient descent; however, their method is still specific to fiber-level modeling of cloth.


\paragraph{Bayesian inference.} A variety of methods used across the sciences are Bayesian in nature; in this paper, we specifically explore Bayesian inference for parameter estimation through Markov chain Monte Carlo sampling of the posterior distribution.

Hamiltonian Monte Carlo (HMC) \cite{Neal2012,Betancourt2017} is an algorithm for sampling a multi-dimensional continuous probability distribution (pdf), given just a piece of code that evaluates the log pdf and its gradient, the method can effectively explore the space, sampling points with probability proportional to the pdf. The gradient information generally leads to much more efficient sampling than with simpler methods such as Metropolis-Hastings.

STAN \cite{Stan} is a software package for Bayesian inference, allowing a user to specify a statistical forward model and parameter, whose posterior can then be sampled using HMC or maximized using a non-linear optimizer (L-BFGS). We use Stan in our results; however, our forward evaluation models are largely implemented as custom Stan functions, using the \torch C++ library. This approach provides us with automatic differentiation and GPU acceleration, while being powerful enough to express a variety of material models.



\section{Preliminaries}
\label{sec:prelim}

\paragraph{Procedural model generation.}
We focus on \emph{procedural material models} which utilize specialized operators to generate spatially varying surface reflectance profiles.
Specifically, let $\btheta$ be the parameters taken by some procedural material generation process $f_0$.
Then, $f_0(\btheta)$ generates the material properties (e.g., albedo, roughness, surface normals, anisotropy, scattering, etc.), in addition to any other parameters required by rendering (e.g. light parameters), which can in turn be converted into a rendered image $\synth$ via the standard rendering process $R$.
This \emph{forward evaluation} process can be summarized as
%
\begin{equation}
	\label{eq:forward}
	\synth = R(f_0(\btheta)) = f(\btheta),
\end{equation}
%
where $f$ is the composition of $R$ and $f_0$.

When modeling real-world materials, it is desirable to capture naturally arising irregularities.
In procedural modeling, this is usually achieved by making the model generation process $f_0$ to take extra random input $\bz$ (e.g., random seeds, pre-generated noise textures, etc.) that is then used to create the irregularities.
This also causes the full forward evaluation to become $f(\btheta; \bz) := R(f_0(\btheta; \bz))$.

\paragraph{Inverse problem specification.}
We consider the problem of inferring procedural model parameters $\btheta$ given a target image $\target$  (which is typically a photograph of a material sample under known illumination).
This, essentially, requires inverting $f$ in Eq.~\eqref{eq:forward}: $\btheta = f^{-1}(\target)$. Direct inversion of $f = R \circ f_0$ is intractable for any but the simplest material and rendering models.
Instead, we aim to find $\btheta$ such that $\synth$ has similar appearance to $\target$:
%
\begin{equation}
	\label{eq:approx}
	\mbox{find} \ \btheta \ \mbox{s.t.} \ \target \approx f(\btheta; \bz),
\end{equation}
%
for some (any) $\bz$, where $\approx$ is an \emph{appearance-match} relation that will be discussed in the next section.

\begin{comment}
Given a target image $\target$ of a material sample (typically a photograph under known illumination, such as flash), our goal is to estimate the vector of material parameters $\btheta$.

To accomplish this, the first component we will need is a forward evaluator $f(\btheta, \bz)$ which (computationally) synthesize an image $\synth$. \sz{Don't we also need $f$ to be given?}
Here we also consider a vector of random parameters $\bz$; these are essentially controlling the features of the image that have no impact on its ``perceived appearance'' but have significant impact on the numerical values of the pixels; for example, the random seeds of the noise functions used, affecting the placement of bumps or scratches. We will require the derivatives of $f$ with respect to $\btheta$ (but generally not $\bz$). \sz{The introduction of $\bz$ looks confusing. Should we simply make $f$ nondeterministic (i.e., a stochastic process)?}

Our goal is to find $\btheta$ such that $\synth$ has similar appearance to $\target$, abstracting away from the randomness of $\bz$:
%
\begin{equation}
\mbox{find} \ \btheta \ \mbox{s.t.} \ \target \approx f(\btheta, \bz),
\end{equation}
%
where $\approx$ is a yet unspecified ``appearance match'' relationship.
\end{comment}

\section{Summary Functions}
\label{sec:summary_func}
%
%In this section, we introduce \emph{summary functions}, our solution to a key challenge in estimating the parameters of procedural material models: the need for a metric capable of comparing a pair of images based only on their overall appearance.
%
%\subsection{Appearance Matching via Summary Functions}
%
To solve the parameter estimation problem using Eq.~\eqref{eq:approx}, a key ingredient is the appearance-match relation.
Unfortunately, we cannot use simplistic image difference metrics such as the L2 or L1 norms.
This is because the features (bumps, scratches, flakes, yarns, etc.) in the images of real-world materials are generally misaligned, even when the two images represent the same material.
In procedural modeling, as shown in Figure \ref{fig:syn1}, with irregularities created differently using $\bz_1$ and $\bz_2$, the same procedural model parameters $\btheta$ can yield slightly different results $f(\btheta; \bz_1)$ and $f(\btheta, \bz_2)$.

\begin{figure}[t]
	\addtolength{\tabcolsep}{-5pt}
	\begin{tabular}{cccc}
		\includegraphics[width=0.24\columnwidth]{images/syn_comp/bump04rd1.jpg} &
		\includegraphics[width=0.24\columnwidth]{images/syn_comp/bump04rd2.jpg} &
		\includegraphics[width=0.24\columnwidth]{images/syn_comp/bump02rd1.jpg} &
		\includegraphics[width=0.24\columnwidth]{images/syn_comp/bump02rd2.jpg} \\
		(a1) & (a2) & (b1) & (b2)
	\end{tabular}
	\caption{\label{fig:syn1}
		Each pair of images among (a, b) are generated using identical model parameters $\btheta$ but different irregularities $\bz$. The pixel-wise L2 norm of the difference between these image pairs is large and not useful for estimating model parameters.
	}
\end{figure}


To address this challenge, we introduce a \emph{summary function}, which abstracts away the unimportant differences in the placement of the features, and summarizes the predicted and target images into smaller vectors whose similarity can be judged with simple metrics like L2 distance.

An image summary function (embedding) $\summ$ is a continuous function that maps an image of a material ($\target$ or $\synth$) into a vector in $\Reals^k$. An idealized summary function would have the property that
%
\begin{equation}
	\summ(f(\btheta_1, \bz_1)) = \summ(f(\btheta_2, \bz_2)) \ \Leftrightarrow \ \btheta_1 = \btheta_2.
\end{equation}
%
That is, applying the summary function would fully abstract away from the randomness $\bz$ and the difference between the two summary vectors would be entirely due to different material properties $\btheta$.

Practical summary functions will satisfy the above only approximately. However, a good practical summary function will embed images of the same material close to each other, and images of different materials further away from each other. Below we discuss several techniques for constructing summary functions.

\subsection{Our Summary Functions}
\label{ssec:example_summary_func}

\paragraph{Statistics of image bins.}
The simplest idea for a summary function is to subdivide the image into $k$ bins (regions) and compute the (scalar or RGB) mean of each region. For mostly isotropic materials, a flash photograph will normally lead to an approximately radially symmetric highlight, where we found concentric bins perform well. For anisotropic highlights (e.g. brushed metal), it makes more sense to define the bins as $k$ vertical or horizontal bands. Adding the standard deviation statistic per bin, in addition to mean, can significantly help with estimating frequencies and size of features such as bumps.

\paragraph{Fourier transforms.}
However, the simple per-bin statistics can be insufficient to precisely match frequency characteristics of material imperfections. A more complex and more powerful tool for summary function design is fast Fourier transforms, which captures frequencies explicitly. These can be applied to the whole image (as a 2D FFT), or part of it, e.g. as 1D FFTs over a subset of rows or columns. Note that automatic computation of derivatives is possible with the FFT, and supported by the PyTorch framework.

\paragraph{Neural summary function.}
Gatys et al. \cite{Gatys2015,Gatys2016} introduced the idea of using the features of an image classification neural network (usually VGG \cite{VGG}) as a descriptor $T_G$ of image texture (or style). Optimizing images to minimize the difference in $T_G$ (combined with other constraints) allowed Gatys et al. to produce impressive, state-of-the art results for parametric texture synthesis and style transfer between images. While further work  has introduced improvements \cite{Risser2017}, we find that the original version from Gatys et al. works already well in our case.

Aittala et al. \cite{Aittala2016} introduced this idea to capturing material parameter textures (albedo, roughness, normal and specular maps) of stationary materials. They optimized for a $256 \times 256$ stationary patch that matches the target image in various crops, using a combination of $T_G$ and a number of special Fourier-domain priors. In our case (for procedural materials), we find that the neural summary function $T_G$ works even more effectively; we can simply apply it to the entire target or simulated images (not requiring crops nor Fourier-domain priors).
%
Specifically, define the Gram matrix $G$ of a set of feature maps $F_1, \cdots, F_n$ as
\begin{equation}
	G = \mbox{mean}(F_i \cdot F_j),
\end{equation}
where the product $F_i \cdot F_j$ is element-wise. $T_G$ is defined as the concatenation of the flattened Gram matrices computed for the feature maps before each pooling operation in VGG19. Note that the size of the Gram matrices depends on the number of feature maps (channels), not their size; thus $T_G$ is independent of input image size.


\section{Bayesian inference for parameter estimation}

In this section, we first summarize the classical non-linear optimization approach to parameter estimation (and inverse problems in general), and its Bayesian formulation as a  maximum a-posteriori (MAP) estimate. These approaches normally provide a single point estimate of the parameter vector. Next, we explain how the Hamiltonian Monte Carlo approach for Bayesian inference extends the classical approach, resolving several important issues.


\subsection{Point estimates}

\paragraph{Non-linear optimization.} As above, assume our goal is to match the target image $\target$, normally the image of a material sample under known illumination conditions. The model has unknown parameters $\btheta$. The \emph{forward evaluator} $f(\btheta, \bz)$ is available as a differentiable subroutine.  Given the availability of an appropriate summary function $\summ$, our goal is to find the value of $\btheta$ whose summary vector fits the summary vector of the target (measurement) $\target$ as closely as possible:
\begin{equation} \label{eq:approx}
	\summ(f(\btheta, \bz)) \approx \summ(\target),
\end{equation}
where we did not specify the meaning of $\approx$ yet. One way to solve this problem in practice is to find an approximate solution for $\btheta$ by choosing a fixed $\bz$ at random and minimizing the nonlinear least squares objective function
\begin{equation}
	\argmin_{\btheta} \|\summ(f(\btheta, \bz)) - \summ(\target)\|^2.
\end{equation}
This can be accomplished by standard non-linear optimization methods like Levenberg-Marquardt or L-BFGS. Regularization is commonly added to improve the stability of this approach, often by adding a term like $\epsilon \|\btheta\|^2$ for a hand-picked constant $\epsilon$, or similar, usually ad-hoc terms.


\paragraph{Maximum a-posteriori estimation.} A technically similar but theoretically cleaner approach to the above problem is to model it in a probabilistic Bayesian framework, as the maximization of the posterior distribution. The key idea is to model the parameters $\btheta$, as well as other quantities needed during estimation, as random variables with corresponding probability distributions.

Specifically, we introduce a \emph{prior} probability distribution $p(\btheta)$ of the parameters, reflecting our pre-existing beliefs about the likely values of the unknown parameters. For example, in most material models, we know what range the color and roughness coefficients of the material should typically be in, and we know that shading normals tend to point up, with some variation.

Furthermore, we model the $\approx$ operator from eq. \ref{eq:approx} as an error distribution. More precisely, we postulate that the difference between the simulated image summary $\summ(f(\btheta, \bz))$ and the target (measured) image summary $\summ(\target)$ follows a known probability distribution; for example, we can use a Gaussian (normal) distribution with zero mean and a $k$-dimensional error vector $\bsigma_e$:
\begin{equation}
	\summ(f(\btheta, \bz)) - \summ(\target) \sim \mathcal N(0, \bsigma_e).
\end{equation}
We find that this error distribution works well in practice, and set $\bsigma_e$ by hand, even though it could be estimated together with the material parameters.

We also have multiple options in handling the random vector $\bz$. While it is certainly theoretically possible to estimate it, we are not really interested in its values;  we find it simpler and more efficient to simply choose $\bz$ randomly, fix it, and assume it known during the process of estimating the ``interesting'' parameters $\btheta$.

Under these assumptions, and after applying the Bayes theorem, we can write down the posterior probability of parameters $\btheta$, conditional on the known values of $\target$ and $\bz$, as:
\begin{equation} \label{eq:posterior}
	p(\btheta | \target, \bz) \propto G(\summ(f(\btheta, \bz)) - \summ(\target); 0, \bsigma_e) \ p(\btheta),
\end{equation}
where the $G$ term is a $k$-dimensional Gaussian with zero mean and standard deviation vector $\bsigma_e$, and $p(\btheta)$ is the prior on the material parameters.
The right side of the equation does not need to be normalized as a pdf; the constant factor is not important in the following.

In the maximum a-posteriori (MAP) framework we estimate the desired parameter values $\btheta$ as the maximum of the posterior pdf $p(\btheta | \target, \bz)$. This problem can be solved using a number of non-linear optimizers. Note the close technical similarity to the classical optimization discussed above: taking the logarithm, the error Gaussian $G$ becomes a quadratic term. Note, if the priors were chosen to be Gaussian as well, the whole problem would turn into a standard non-linear least squares problem. In practice, because we do make use of non-Gaussian priors, and because our $\bsigma_e$ can be a function of other values, we have a general non-linear log-posterior; this is not an issue and can be handled effectively.

In summary, unlike the classical optimization approach, where the goal is to find a solution to the approximate equations $f(\btheta, \bz) \approx \target$ or (in the summary function framework) $\summ(f(\btheta, \bz)) \approx \summ(\target)$, in the MAP framework our goal becomes finding the maximum of the \emph{posterior probability distribution} of $\btheta$: this is the conditional probability $p(\btheta | \target, \bz)$ defined in eq. \ref{eq:posterior}.



\subsection{Monte Carlo sampling of the posterior}

The point estimate approach gives satisfactory results in some cases, but also raises several questions. For example, will the optimization converge? Assuming the algorithm converges, is the resulting minimum global or just local? If local, how many other local minima are there? Even if the minimum found is global, does it truly give the best solution to the original problem, or is the solution biased by our choice of priors or summary function?

Furthermore, there could be an entire subset of the parameter space giving solutions of approximately equivalent quality; in other words, the solution space can exhibit a \emph{similarity structure}. This problem becomes important with more powerful models containing more parameters. Frequently, this issue would historically lead computer graphics researchers to avoid additional power in models, instead preferring simple BRDFs.

In this paper, we use the well-known technique of full Bayesian inference, sampling the posterior pdf defined in eq. \ref{eq:posterior} using Markov chain techniques, specifically Hamiltonian Monte Carlo \cite{Betancourt2017}. While well explored in statistics and various scientific fields, to our knowledge this technique has not been used for material capture in computer graphics.

The goal of the sampling is to explore the posterior with many (typically thousands or more) samples, each of which represents a material parameter vector consistent with the target image. Plotting these samples projected into two dimensions (for a given pair of parameters) gives valuable insight into similarity structures. Furthermore, interactively clicking on samples and observing the predicted result can help a user to quickly zoom in on a preferred solution, which an automatic optimization algorithm is fundamentally incapable of.

Technically, HMC is similar to Markov chain methods such as Metropolis-Hastings, but is much more efficient for posterior sampling problems in cases where the derivatives of the model are available (always true in our case, as all our forward evaluators are implemented in \torch). We use the Stan framework to define the top-level model (including parameter priors $p(\btheta)$ and the error vector $\bsigma_e$), and to carry out the posterior sampling.



% - the approximation error inherent in the problem

% - an uninformative prior (one where all parameter values are equally likely)
% - this gives rise to a \emph{maximum likelihood estimate}

\section{Material Models and Results}
\label{sec:results}
%
%we present specific material models and parameter estimation results using our Bayesian inference framework with summary functions.
We now demonstrate the effectiveness of our technique by fitting six procedural material models---bumpy microfacet surface, brushed metal, metallic paint with flakes, leather, plaster, and wood---to a mix of synthetic and real target images.
We also show a translucent material in the supplementary material.

Our forward evaluation process has the camera and light co-located.
This configuration closely matches a mobile phone camera with flash (which is what we use to take the real target images) and simplifies some BRDF formulations (because the incoming, outgoing, and half-way vectors are all identical).
Further, we assume that the distance between camera and sample is known as it is generally easy to measure or estimate.
The knowledge of the camera field of view allows us to compute the physical scale of the resulting pixels.
Lastly, we treat light intensity and vignetting (expressed as an image-space Gaussian function) as (unknown) parameters of the forward evaluation process so that they do not need to be calibrated.

%We use real-world units (centimeters) for all relevant parameters; this ensures that the resulting materials have physical proportions. We model the light intensity as an unknown, thus we do not require any calibration procedures. Finally, we observed that the vignetting from the cell phone camera has an impact on the results. While we could post-process the images to counter the effect, we find it easier and more appropriate within our framework to simply model the vignetting as a broad Gaussian, whose standard deviation becomes yet another parameter. Photographs are taken with an iPhone~7. For some materials, overexposure is unavoidable; we simply let overly bright areas clamp, and apply the same clamping to our forward simulation.

All the procedural material models we used, which will be detailed in \S\ref{ssec:proc_models}, are implemented using PyTorch which %using array-level operations; this 
automatically provides GPU acceleration and computes derivatives through backpropagation. %and lets us express fairly complex operations, including microfacet BRDF evaluation, fast Fourier transforms, texture queries, color operations, and more. 
%The GPU we use is an Nvidia GTX 1080. 
For all material parameter inference tasks, our forward evaluation generates $256 \times 256$ images.
Notice that the recovered parameters can then be used to generate results with much higher resolution because the procedural models are generally resolution-independent.
%The results are visualized at higher resolutions, since the procedural materials allow for resolution independence; there is no requirement to use the resolution used for parameter estimation also in final rendering.

We show results generated using six synthetic images in Figure~\ref{fig:synth} and four real photographs (taken with an iPhone~7) in Figure~\ref{fig:real}.
%The captions of the figures provide more detail. 
Please see the supplemental material for more results, including animations illustrating the optimization and sampling progress.

%\begin{figure*}[t]
%	\addtolength{\tabcolsep}{-4.5pt}
%	\begin{tabular}{ccccccccc}
%		& \multicolumn{2}{c}{\toptext{2\resultwidth}{Point estimate}} & \multicolumn{5}{c}{\toptext{5\resultwidth}{Bayesian inference}}\\[-4pt]
%		target & loss & optimize & posterior & sample-1 & sample-2 & sample-3& sample-4
%		\\
%		\includegraphics[width=\resultwidth]{images/synth/bump/out/target.jpg} &
%		\includegraphics[width=\resultwidth]{images/synth/bump/out/loss.pdf} &
%		\includegraphics[width=\resultwidth]{images/synth/bump/out/optim.jpg} &
%		\includegraphics[width=\resultwidth]{images/synth/bump/out/posterior.pdf} &
%		\includegraphics[width=\resultwidth]{images/synth/bump/out/good1.jpg} &
%		\includegraphics[width=\resultwidth]{images/synth/bump/out/good2.jpg} &
%		\includegraphics[width=\resultwidth]{images/synth/bump/out/good3.jpg} &
%		\includegraphics[width=\resultwidth]{images/synth/bump/out/bad1.jpg}
%		\\
%		\includegraphics[width=\resultwidth]{images/synth/leather/out/target.jpg} &
%		\includegraphics[width=\resultwidth]{images/synth/leather/out/loss.pdf} &
%		\includegraphics[width=\resultwidth]{images/synth/leather/out/optim.jpg} &
%		\includegraphics[width=\resultwidth]{images/synth/leather/out/posterior.pdf} &
%		\includegraphics[width=\resultwidth]{images/synth/leather/out/good1.jpg} &
%		\includegraphics[width=\resultwidth]{images/synth/leather/out/good2.jpg} &
%		\includegraphics[width=\resultwidth]{images/synth/leather/out/good3.jpg} &
%		\includegraphics[width=\resultwidth]{images/synth/leather/out/bad1.jpg}
%		\\
%		\includegraphics[width=\resultwidth]{images/synth/plaster/out/target.jpg} &
%		\includegraphics[width=\resultwidth]{images/synth/plaster/out/loss.pdf} &
%		\includegraphics[width=\resultwidth]{images/synth/plaster/out/optim.jpg} &
%		\includegraphics[width=\resultwidth]{images/synth/plaster/out/posterior.pdf} &
%		\includegraphics[width=\resultwidth]{images/synth/plaster/out/good1.jpg} &
%		\includegraphics[width=\resultwidth]{images/synth/plaster/out/good2.jpg} &
%		\includegraphics[width=\resultwidth]{images/synth/plaster/out/good3.jpg} &
%		\includegraphics[width=\resultwidth]{images/synth/plaster/out/bad1.jpg}
%		\\
%		\includegraphics[width=\resultwidth]{images/synth/flake/out/target.jpg} &
%		\includegraphics[width=\resultwidth]{images/synth/flake/out/loss.pdf} &
%		\includegraphics[width=\resultwidth]{images/synth/flake/out/optim.jpg} &
%		\includegraphics[width=\resultwidth]{images/synth/flake/out/posterior.pdf} &
%		\includegraphics[width=\resultwidth]{images/synth/flake/out/good1.jpg} &
%		\includegraphics[width=\resultwidth]{images/synth/flake/out/good2.jpg} &
%		\includegraphics[width=\resultwidth]{images/synth/flake/out/good3.jpg} &
%		\includegraphics[width=\resultwidth]{images/synth/flake/out/bad1.jpg}
%		\\
%		\includegraphics[width=\resultwidth]{images/synth/metal/out/target.jpg} &
%		\includegraphics[width=\resultwidth]{images/synth/metal/out/loss.pdf} &
%		\includegraphics[width=\resultwidth]{images/synth/metal/out/optim.jpg} &
%		\includegraphics[width=\resultwidth]{images/synth/metal/out/posterior.pdf} &
%		\includegraphics[width=\resultwidth]{images/synth/metal/out/good1.jpg} &
%		\includegraphics[width=\resultwidth]{images/synth/metal/out/good2.jpg} &
%		\includegraphics[width=\resultwidth]{images/synth/metal/out/good3.jpg} &
%		\includegraphics[width=\resultwidth]{images/synth/metal/out/bad1.jpg}
%		\\
%		\includegraphics[width=\resultwidth]{images/synth/wood/out/target.jpg} &
%		\includegraphics[width=\resultwidth]{images/synth/wood/out/loss.pdf} &
%		\includegraphics[width=\resultwidth]{images/synth/wood/out/optim.jpg} &
%		\includegraphics[width=\resultwidth]{images/synth/wood/out/posterior.pdf} &
%		\includegraphics[width=\resultwidth]{images/synth/wood/out/good1.jpg} &
%		\includegraphics[width=\resultwidth]{images/synth/wood/out/good2.jpg} &
%		\includegraphics[width=\resultwidth]{images/synth/wood/out/good3.jpg} &
%		\includegraphics[width=\resultwidth]{images/synth/wood/out/bad1.jpg}
%		\\
%		& & \qquad \qquad \, low &
%		\includegraphics[width=\resultwidthpdf]{images/img/colorbar.jpg} &
%		high \qquad \qquad \,& & &
%	\end{tabular}
%	%
%	\caption{\label{fig:synth}
%		\textbf{Optimization and HMC sampling on synthetic images.} Each row corresponds to a different material. From top: bump, leather, plaster, metallic flake, brushed metal and wood. Column 1 is rendered images using different forward models. We show optimization results in columns 2 and 3, samplings in the rest columns. For high dimensional posterior visualization, we project them to 2D using PCA. Here we only show the first two major components. The three red dots corresponding to sample-1,2,3, which are closer to the peak of high dimensional distribution. and the green dot (sample-4) is the opposite. More results please refer to supplemental materials.
%	}
%\end{figure*}

\begin{figure*}[t]
	\centering
	\addtolength{\tabcolsep}{-4.5pt}
	\begin{tabular}{ccccccccc}
		target & sample-1 & sample-2 & sample-3 & & target & sample-1 & sample-2 & sample-3
		\\
		\begin{overpic}[width=\resultwidth]{images/synth/bump/out/target.jpg}
			\imglabel{Bump-1}
		\end{overpic} &
		\includegraphics[width=\resultwidth]{images/synth/bump/out/good1.jpg} &
		\includegraphics[width=\resultwidth]{images/synth/bump/out/good2.jpg} &
		\includegraphics[width=\resultwidth]{images/synth/bump/out/bad1.jpg} &
		&
		\begin{overpic}[width=\resultwidth]{images/synth/bump_2/out/target.png}
			\imglabel{Bump-2}
		\end{overpic} &
		\includegraphics[width=\resultwidth]{images/synth/bump_2/out/good1.png} &
		\includegraphics[width=\resultwidth]{images/synth/bump_2/out/good2.png} &
		\includegraphics[width=\resultwidth]{images/synth/bump_2/out/bad1.png}
		\\
		\begin{overpic}[width=\resultwidth]{images/synth/leather/out/target.jpg}
			\imglabel{Leather-1}
		\end{overpic} &
		\includegraphics[width=\resultwidth]{images/synth/leather/out/good1.jpg} &
		\includegraphics[width=\resultwidth]{images/synth/leather/out/good2.jpg} &
		\includegraphics[width=\resultwidth]{images/synth/leather/out/bad1.jpg} &
		&
		\begin{overpic}[width=\resultwidth]{images/synth/leather_2/out/target.png}
			\imglabel{Leather-2}
		\end{overpic} &
		\includegraphics[width=\resultwidth]{images/synth/leather_2/out/good1.png} &
		\includegraphics[width=\resultwidth]{images/synth/leather_2/out/good2.png} &
		\includegraphics[width=\resultwidth]{images/synth/leather_2/out/bad1.png}
		\\
		\begin{overpic}[width=\resultwidth]{images/synth/plaster/out/target.jpg}
			\imglabel{Plaster-1}
		\end{overpic} &
		\includegraphics[width=\resultwidth]{images/synth/plaster/out/good1.jpg} &
		\includegraphics[width=\resultwidth]{images/synth/plaster/out/good2.jpg} &
		\includegraphics[width=\resultwidth]{images/synth/plaster/out/bad1.jpg} &
		&
		\begin{overpic}[width=\resultwidth]{images/synth/plaster_2/out/target.png}
			\imglabel{Plaster-2}
		\end{overpic} &
		\includegraphics[width=\resultwidth]{images/synth/plaster_2/out/good1.png} &
		\includegraphics[width=\resultwidth]{images/synth/plaster_2/out/good2.png} &
		\includegraphics[width=\resultwidth]{images/synth/plaster_2/out/bad1.png}
		\\
		\begin{overpic}[width=\resultwidth]{images/synth/flake/out/target.jpg}
			\imglabel{Metallicflake-1}
		\end{overpic} &
		\includegraphics[width=\resultwidth]{images/synth/flake/out/good1.jpg} &
		\includegraphics[width=\resultwidth]{images/synth/flake/out/good2.jpg} &
		\includegraphics[width=\resultwidth]{images/synth/flake/out/bad1.jpg} &
		&
		\begin{overpic}[width=\resultwidth]{images/synth/flake_2/out/target.png}
			\imglabel{Metallicflake-2}
		\end{overpic} &
		\includegraphics[width=\resultwidth]{images/synth/flake_2/out/good1.png} &
		\includegraphics[width=\resultwidth]{images/synth/flake_2/out/good2.png} &
		\includegraphics[width=\resultwidth]{images/synth/flake_2/out/bad1.png}
		\\
		\begin{overpic}[width=\resultwidth]{images/synth/metal/out/target.jpg}
			\imglabel{Brushmetal-1}
		\end{overpic} &
		\includegraphics[width=\resultwidth]{images/synth/metal/out/good1.jpg} &
		\includegraphics[width=\resultwidth]{images/synth/metal/out/good2.jpg} &
		\includegraphics[width=\resultwidth]{images/synth/metal/out/bad1.jpg} &
		&
		\begin{overpic}[width=\resultwidth]{images/synth/metal_2/out/target.png}
			\imglabel{Brushmetal-2}
		\end{overpic} &
		\includegraphics[width=\resultwidth]{images/synth/metal_2/out/good1.png} &
		\includegraphics[width=\resultwidth]{images/synth/metal_2/out/good2.png} &
		\includegraphics[width=\resultwidth]{images/synth/metal_2/out/bad1.png}
		\\
		\begin{overpic}[width=\resultwidth]{images/synth/wood/out/target.jpg}
			\imglabel{Wood-1}
		\end{overpic} &
		\includegraphics[width=\resultwidth]{images/synth/wood/out/good1.jpg} &
		\includegraphics[width=\resultwidth]{images/synth/wood/out/good2.jpg} &
		\includegraphics[width=\resultwidth]{images/synth/wood/out/bad1.jpg} & &
		\begin{overpic}[width=\resultwidth]{images/synth/wood_2/out/target.png}
			\imglabel{Wood-2}
		\end{overpic} &
		\includegraphics[width=\resultwidth]{images/synth/wood_2/out/good1.png} &
		\includegraphics[width=\resultwidth]{images/synth/wood_2/out/good2.png} &
		\includegraphics[width=\resultwidth]{images/synth/wood_2/out/bad1.png}
	\end{tabular}
	%
	\caption{\label{fig:synth}
		Results of our MCMC sampling on \textbf{synthetic} inputs. Each row corresponds to two examples of a different material model. For each example, the first column is the synthetic target image. We show MCMC samples in the other columns, where sample-1 and sample-2 are chosen closer to the peak of the posterior distribution, and sample-3 is further away. More results please refer to supplemental materials.
	}
\end{figure*}

\begin{figure}[t]
	\centering
	\addtolength{\tabcolsep}{-4.5pt}
	\begin{tabular}{cccc}
		target & sample-1 & sample-2 & sample-3
		\\
		\begin{overpic}[width=\resultwidth]{images/synth/leather_3/out/target.png}
			\imglabel{Leather-1}
		\end{overpic} &
		\includegraphics[width=\resultwidth]{images/synth/leather_3/out/good1.png} &
		\includegraphics[width=\resultwidth]{images/synth/leather_3/out/bad1.png} &
		\includegraphics[width=\resultwidth]{images/synth/leather_3/out/bad2.png}
		\\
		&
		\begin{overpic}[width=\resultwidth]{images/synth/cell/cell_1.png}
			\put(0,0){\color{green}%
				\frame{\includegraphics[width=0.4\resultwidth]{images/synth/cell/cell_1_zoom.png}}}
		\end{overpic}
		&
		\begin{overpic}[width=\resultwidth]{images/synth/cell/cell_2.png}
			\put(0,0){\color{green}%
				\frame{\includegraphics[width=0.4\resultwidth]{images/synth/cell/cell_2_zoom.png}}}
		\end{overpic}
		&
		\begin{overpic}[width=\resultwidth]{images/synth/cell/cell_3.png}
			\put(0,0){\color{green}%
				\frame{\includegraphics[width=0.4\resultwidth]{images/synth/cell/cell_3_zoom.png}}}
		\end{overpic}
		\\
		\multicolumn{4}{c}{(a)}
		\\
		\begin{overpic}[width=\resultwidth]{images/synth/plaster_2/out/target.png}
			\imglabel{Plaster-2}
		\end{overpic} &
		\includegraphics[width=\resultwidth]{images/synth/plaster_2/out/good1.png} &
		\includegraphics[width=\resultwidth]{images/synth/plaster_2/out/bad2.png} &
		\includegraphics[width=\resultwidth]{images/synth/plaster_2/out/bad3.png}
		\\
		&
		\begin{overpic}[width=\resultwidth]{images/synth/noise/noise_1.png}
			\put(0,0){\color{green}%
				\frame{\includegraphics[width=0.4\resultwidth]{images/synth/noise/noise_1_zoom.png}}}
		\end{overpic}
		&
		\begin{overpic}[width=\resultwidth]{images/synth/noise/noise_2.png}
			\put(0,0){\color{green}%
				\frame{\includegraphics[width=0.4\resultwidth]{images/synth/noise/noise_2_zoom.png}}}
		\end{overpic}
		&
		\begin{overpic}[width=\resultwidth]{images/synth/noise/noise_3.png}
			\put(0,0){\color{green}%
				\frame{\includegraphics[width=0.4\resultwidth]{images/synth/noise/noise_3_zoom.png}}}
		\end{overpic}
		\\
		\multicolumn{4}{c}{(b)}
	\end{tabular}
	%
	\caption{\label{fig:discrete}
		\textbf{MCMC sampling with discrete parameters.} In these examples, we illustrate the ability of our sampling to handle discrete parameters. In both examples, one noise inputs used in the procedural model can be switched between several different types of noise. Out of the thousands of sampled solutions, we pick three that have different settings of the discrete parameter where the (log) pdf values decrease from sample-1 to sample-3.
	}
\end{figure}

\begin{figure*}[t]
	\addtolength{\tabcolsep}{-4.5pt}
	\begin{tabular}{cccccccc}
		& \multicolumn{2}{c}{\toptext{2\resultwidth}{Point estimate}} & \multicolumn{5}{c}{\toptext{5\resultwidth}{Bayesian inference}}\\[-4pt]
		target & loss & optimize & posterior & sample-1 & sample-2 & sample-3& sample-4
		\\
		\includegraphics[width=\resultwidth]{images/real/bump/out/target.jpg} &
		\includegraphics[width=\resultwidth]{images/real/bump/out/loss.pdf} &
		\includegraphics[width=\resultwidth]{images/real/bump/out/optim.jpg} &
		\includegraphics[width=\resultwidth]{images/real/bump/out/posterior.pdf} &
		\includegraphics[width=\resultwidth]{images/real/bump/out/good1.jpg} &
		\includegraphics[width=\resultwidth]{images/real/bump/out/good2.jpg} &
		\includegraphics[width=\resultwidth]{images/real/bump/out/good3.jpg} &
		\includegraphics[width=\resultwidth]{images/real/bump/out/bad1.jpg}
		\\
		\includegraphics[width=\resultwidth]{images/real/leather/out/target.jpg} &
		\includegraphics[width=\resultwidth]{images/real/leather/out/loss.pdf} &
		\includegraphics[width=\resultwidth]{images/real/leather/out/optim.jpg} &
		\includegraphics[width=\resultwidth]{images/real/leather/out/posterior.pdf} &
		\includegraphics[width=\resultwidth]{images/real/leather/out/good1.jpg} &
		\includegraphics[width=\resultwidth]{images/real/leather/out/good2.jpg} &
		\includegraphics[width=\resultwidth]{images/real/leather/out/good3.jpg} &
		\includegraphics[width=\resultwidth]{images/real/leather/out/bad1.jpg}
		\\
		\includegraphics[width=\resultwidth]{images/real/plaster/out/target.jpg} &
		\includegraphics[width=\resultwidth]{images/real/plaster/out/loss.pdf} &
		\includegraphics[width=\resultwidth]{images/real/plaster/out/optim.jpg} &
		\includegraphics[width=\resultwidth]{images/real/plaster/out/posterior.pdf} &
		\includegraphics[width=\resultwidth]{images/real/plaster/out/good1.jpg} &
		\includegraphics[width=\resultwidth]{images/real/plaster/out/good2.jpg} &
		\includegraphics[width=\resultwidth]{images/real/plaster/out/good3.jpg} &
		\includegraphics[width=\resultwidth]{images/real/plaster/out/bad1.jpg}
		\\
%		\includegraphics[width=\resultwidth]{images/real/flake/out/target.jpg} &
%		\includegraphics[width=\resultwidth]{placeholder/placeholder3.jpg} &
%		\includegraphics[width=\resultwidth]{images/real/flake/out/optim.jpg} &
%		\includegraphics[width=\resultwidth]{placeholder/placeholder3.jpg} &
%		\includegraphics[width=\resultwidth]{images/real/flake/out/good1.jpg} &
%		\includegraphics[width=\resultwidth]{images/real/flake/out/good2.jpg} &
%		\includegraphics[width=\resultwidth]{images/real/flake/out/good3.jpg} &
%		\includegraphics[width=\resultwidth]{placeholder/placeholder3.jpg}
%		\\
%		\includegraphics[width=\resultwidth]{images/real/metal/out/target.jpg} &
%		\includegraphics[width=\resultwidth]{placeholder/placeholder3.jpg} &
%		\includegraphics[width=\resultwidth]{placeholder/placeholder3.jpg} &
%		\includegraphics[width=\resultwidth]{placeholder/placeholder3.jpg} &
%		\includegraphics[width=\resultwidth]{images/real/metal/out/good1.jpg} &
%		\includegraphics[width=\resultwidth]{images/real/metal/out/good2.jpg} &
%		\includegraphics[width=\resultwidth]{images/real/metal/out/good3.jpg} &
%		\includegraphics[width=\resultwidth]{placeholder/placeholder3.jpg}
%		\\
		\includegraphics[width=\resultwidth]{images/real/wood/out/target.jpg} &
		\includegraphics[width=\resultwidth]{images/real/wood/out/loss.pdf} &
		\includegraphics[width=\resultwidth]{images/real/wood/out/optim.jpg} &
		\includegraphics[width=\resultwidth]{images/real/wood/out/posterior.pdf} &
		\includegraphics[width=\resultwidth]{images/real/wood/out/good1.jpg} &
		\includegraphics[width=\resultwidth]{images/real/wood/out/good2.jpg} &
		\includegraphics[width=\resultwidth]{images/real/wood/out/good3.jpg} &
		\includegraphics[width=\resultwidth]{images/real/wood/out/bad1.jpg}
		\\
		& & \qquad \qquad \, low &
		\includegraphics[width=\resultwidthpdf]{images/img/colorbar.png} &
		high \qquad \qquad \,& & &
	\end{tabular}
	\caption{\label{fig:real}
		\textbf{Optimization and HMC sampling on real photos.}
		Each row corresponds to a different material. From top: bump, leather, plaster, and wood. Similar to Figure~\protect\ref{fig:synth}, except column 1 here contains real photos, columns 2 and 3 show point estimates (via non-linear optimization), and the remaining columns show HMC sampling results. Please refer to the supplemental material to see more results.
	}
\end{figure*}

\begin{table}[t]
	\centering
	\caption{\label{fig:performance}
		%\sz{Need to be updated if we used MALA.}
		Performance statistics for our MCMC-based posterior sampling.
		The numbers are collected using a workstation equipped with an Intel i7-6800K six-core CPU and an Nvidia GTX 1080 GPU.  %\protect\footnotemark.
	}
	\begin{tabular}{l|c|c}
		\multirow{2}{*}{\textbf{Material}} & \multirow{2}{*}{\textbf{\# params}} & {\textbf{MCMC}}\\
		& & (1k iter.)\\
		\hline
		Bump    &  8 & 180 s\\
		Leather & 12 & 194 s\\
		Plaster & 11 & 190 s\\
		Flakes  & 13 & 187 s\\
		Metal   & 10 & 182 s\\
		Wood    & 23 & 290 s
	\end{tabular}
\end{table}

%\footnotetext{In HMC sampling, each sample needs $(s+3)/r$ forward evaluations on average where $s$ indicates the number of leapfrog steps and $r$ denotes the acceptance rate. In practice, we set $s = 4$ and have $r = 70\%$, causing the expected number of evaluations per sample to be 10.}


\subsection{Procedural Material  Models}
\label{ssec:proc_models}
%
We now describe six procedural models tested. Please refer to the supplement for their PyTorch implementation.

\paragraph{Bumpy microfacet surface.}
This model depicts an opaque dielectric surface with an isotropic noise heightfield. We use a standard microfacet BRDF with the GGX normal distribution~\cite{Walter2007} combined with a normal map computed from an explicitly constructed heightfield. We assume that the Fresnel reflectance at normal incidence can be computed from a known index of refraction (a value of 1.5 is a good estimate for plastics). We assume an unknown roughness $r$ (GGX parameter $\alpha=r^2$) and a Lambertian diffuse term with unknown albedo $\rho$. This model is identical to Wang et al.~\cite{Wang2011}, except using the GGX instead of Beckmann microfacet distribution. The main practical difference from the capture setup in that paper is that we use a point light, instead of step-edge illumination.

The bumpy heightfield is constructed using an inverse Fourier process including: (i)~choosing a power spectrum in the continuous Fourier domain; (ii)~discretizing it onto a grid of complex numbers; (iii)~randomly choosing the phase of each texel on the grid (while keeping the chosen amplitude); and (iv)~applying an inverse fast Fourier transform whose 
real component becomes the resulting heightfield. 
At render time, we use the normal map derived from this heightfield.
%The normal map can be computed from the heightfield (finite differences work well).

\paragraph{Leather and plaster.}
These materials can be modeled similarly as the aforementioned bumpy surfaces except for the computation of the heightfield and roughness.
For plaster, a fractal noise texture is scaled (in space and intensity) and thresholded (controlled by additional parameters) to produce both the heightfield and a roughness variation texture. For leather, on the contrary, a Voronoi cell map is used to get the effect of leather-like cells (with parameters for scaling and thresholding), and additional small-scale fractal noise is added.

\paragraph{Brushed metal.} The brushed metal material extends the above bumpy surface, by introducing anisotropy to both the GGX normal distribution and the noise heightfield used to compute the normal map, while dropping the diffuse term. We make both the BRDF and the Fourier-domain Gaussian power spectrum anisotropic. The parameters of the model thus include two roughnesses, as well as two Fourier-domain standard deviations.  We make the anisotropic highlight vertical and centered in the target image.

\paragraph{Metallic flakes.} Metallic paint with flakes is a stochastic material with multiple BRDF lobes (caused by light reflecting off the flakes). Our model involves three components, each being an isotropic microfacet lobe, to describe top coating, flakes and glow, respectively. The top coating is usually highly specular, and we make its roughness a model parameter. We assume an index of refraction of 1.5, implying a Fresnel (Schlick) reflectivity at normal incidence of 0.04. The flakes are chosen as Voronoi cells of a random blue-noise point distribution; they have a roughness parameter and varying normals chosen from the Beckmann distribution with an unknown roughness, and with unknown Fresnel reflectivity. The scale of the cell map is itself a (differentiable) parameter. Lastly, the glow is a component approximating the internal scattering between the top interface and the flakes, and has its own roughness, Fresnel reflectivity and a flat normal. An extra weight parameter linearly combines the flakes and the glow.

\paragraph{Wood.} Lastly, we created a partial PyTorch implementation of the comprehensive 3D wood model of Liu et al.~\cite{Liu2016}. This material is a 3D model of the growth rings of a tree, with a number of parameters controlling colors and widths of growth rings, as well as global distortions and small-scale noise features. We do not implement pores or anisotropic fiber highlights. The 3D wood is finally projected by a cutting plane to image space, defining diffuse albedo, roughness and height.

\section{Conclusion}
\label{sec:conclusion}
%
%We have introduced a differentiable Bayesian framework for parameter estimation of procedural material models demonstrating various phenomena and optical properties. Procedurals have been gaining significant traction in the industry, because they can cover large areas without repetition, and are easily editable.
%We introduced a differentiable forward evaluator, implemented using PyTorch. We proposed several \emph{summary functions}, enabling us to compare a synthetic simulation image to a target image (photo) effectively, by comparing their summary vectors. We have shown that a neural summary function \cite{Gatys2015,Aittala2016} works well in the procedural material setting, and generally outperforms classical summary functions.
%
Procedural material models have become increasingly more popular in the industry, thanks to their flexibility, compactness, as well as easy editability.
In this paper, we introduced a new computational framework to solve the inverse problem: the inference of procedural model parameters based on a single input image.

The first major ingredient to our technique is a family of \emph{summary functions}, from hand-crafted to neural-network based~\cite{Gatys2015,Aittala2016}, that enable robust calculation of image differences (without requiring pixel-level alignments). The second ingredient is a \emph{Bayesian inference method} that leverages Hamiltonian Monte Carlo (HMC) to sample posterior distributions of procedural material parameters.  This technique provides users additional information beyond single point estimates and, to our knowledge, has previously not been applied to inverse-rendering problems.

In the future, we would like to increase the complexity of the models supported even further, to handle materials like woven fabrics, transmissive BTDFs, and more.
%-------------------------------------------------------------------------
% bibtex
\bibliographystyle{eg-alpha-doi} 
\bibliography{references}       

% biblatex with biber
% \printbibliography                

\end{document}
