Procedural material models have been gaining traction in many applications thanks to their flexibility, compactness, and easy editability.
We explore the inverse rendering problem of procedural material parameter estimation from photographs, presenting a unified view of the problem in a Bayesian framework.
In addition to computing point estimates of the parameters by optimization, our framework uses a Markov Chain Monte Carlo approach to sample the space of plausible material parameters, providing a collection of plausible matches that a user can choose from, and efficiently handling both discrete and continuous model parameters.
To demonstrate the effectiveness of our framework, we fit procedural models of a range of materials---wall plaster, leather, wood, anisotropic brushed metals and layered metallic paints---to both synthetic and real target images.
